% !TEX TS-program = pdflatex
% !TEX encoding = UTF-8 Unicode

% This is a simple template for a LaTeX document using the "article" class.
% See "book", "report", "letter" for other types of document.

\documentclass[11pt]{article} % use larger type; default would be 10pt

\usepackage[utf8]{inputenc} % set input encoding (not needed with XeLaTeX)

%%% Examples of Article customizations
% These packages are optional, depending whether you want the features they provide.
% See the LaTeX Companion or other references for full information.

%%% PAGE DIMENSIONS\\
\usepackage[top=0.2in, bottom=0.5in, left=0.7in, right=0.70in]{geometry}


 % to change the page dimensions
\geometry{a4paper} % or letterpaper (US) or a5paper or....

% \geometry{margin=1.2in} % for example, change the margins to 2 inches all round
% \geometry{landscape} % set up the page for landscape
%   read geometry.pdf for detailed page layout information

\usepackage{graphicx} % support the \includegraphics command and options

% \usepackage[parfill]{parskip} % Activate to begin paragraphs with an empty line rather than an indent

%%% PACKAGES
\usepackage{booktabs} % for much better looking tables
\usepackage{array} % for better arrays (eg matrices) in maths
\usepackage{paralist} % very flexible & customisable lists (eg. enumerate/itemize, etc.)
\usepackage{verbatim} % adds environment for commenting out blocks of text & for better verbatim
\usepackage{subfig} % make it possible to include more than one captioned figure/table in a single float
% These packages are all incorporated in the memoir class to one degree or another...

%%% HEADERS & FOOTERS
\usepackage{fancyhdr} % This should be set AFTER setting up the page geometry
\pagestyle{fancy} % options: empty , plain , fancy
\renewcommand{\headrulewidth}{0pt} % customise the layout...
\lhead{}\chead{}\rhead{}
\lfoot{}\cfoot{\thepage}\rfoot{}

%%% SECTION TITLE APPEARANCE
%\usepackage{sectsty}
%\allsectionsfont{\sffamily\mdseries\upshape} % (See the fntguide.pdf for font help)
% (This matches ConTeXt defaults)

%%% ToC (table of contents) APPEARANCE
\usepackage[nottoc,notlof,notlot]{tocbibind} % Put the bibliography in the ToC
\usepackage[titles,subfigure]{tocloft} % Alter the style of the Table of Contents
\renewcommand{\cftsecfont}{\rmfamily\mdseries\upshape}
\renewcommand{\cftsecpagefont}{\rmfamily\mdseries\upshape} % No bold!
\usepackage{titling}
\newcommand{\subtitle}[1]{%
  \posttitle{%
    \par\end{center}
    \begin{center}\large#1\end{center}
    \vskip0.5em}%
}
%%% END Article customizations

%%% The "real" document content comes below...

\title{\bf{CS 335: Compiler Design Course Project}\\
Ada Version 3\\
Group 12\\}
\subtitle{Guide: Prof. S. K. Aggarwal\\Mentor TA: Ajay Kumar}
\author{
Ashudeep Singh\\
10162
\and
Chandra Prakash\\
10209
\and 
Deepak Pathak\\
10222
\and
Kaustubh Tapi\\
10346
}
\date{} % Activate to display a given date or no date (if empty),
         % otherwise the current date is printed 

\begin{document}
\maketitle

\section{Aim of the project}
To build a compiler for Ada Programming Language supporting basic arithmetic and relational operations, conditional statements, iterative statements, procedures, functions, object-oriented programming features like classes with inheritance and polymorphism. Error handling and type checking are also to be handled.
\section{What has been achieved}
We have implemented the following features of Ada language using PLY (Python-Lex-Yacc) 
\begin{itemize}
\item	basic data types
\item	arithmetic and relational operators
\item	if-then-else statements
\item	iterative statements (for and while loops)
\item	one dimensional and multi-dimensional arrays
\item	procedures, recursive procedures(self and mutual), nested procedures
\begin{itemize}
\item Call by value
\item Call by reference.
\end{itemize}
%\item	objects : attributes, functions, inheritance, access modifiers
\item Simple Classes (i.e. packages) with procedures (Nesting in packages is not handled)
\item	error handling and type checking
%\item	semantic analysis of case statements
\item	Printing outputs as integers and characters.
\item	Generation of 3 address code and corresponding SPIM code (Standard short circuiting approach has been implemented using back-patching for 3-address code generation)

\end{itemize}
\section{What has not been achieved}
\begin{itemize}
\item Case conditional statements
\item Object oriented features like polymorphism, inheritance etc
\item Access types
\item Abstract types, interfaces and over-riding
\end{itemize}
\section*{Contributions}
All the group members have contributed equally (25\% each) to the project.

\end{document}